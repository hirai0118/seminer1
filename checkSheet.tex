% *******************************************************************************
% The command of check list
% Arguments are 0, 1 and 2.
% 0 is a blank. 1 is a check mark. 2 is a hyphen.
% *******************************************************************************
\newcommand{\listcheck}[4]{
  \ifcase #1 &\or
  $\surd$& \or --&\fi
  \ifcase #2 &\or
  $\surd$& \or --&\fi
  \ifcase #3 &\or
  $\surd$& \or --&\fi
  \ifcase #4 \or
  $\surd$ \or --\fi
}
% *******************************************************************************
\begin{table*}[htbp]
  \centering
  \caption{The referee of seminar document and power point.}\label{tab.check}
  \begin{tabular}{|c|c|c||c|c|c|}\hline
    \makebox[30pt][c]{Number}&\makebox[25pt][c]{Grade}&\makebox[155pt][c]{Name}&
    \makebox[30pt][c]{Number}&\makebox[25pt][c]{Grade}&\makebox[155pt][c]{Name}\\\hline

    % number&grade&name&&
    % number&grade&name\\\hline
    1 & & & 4 &  & \\\hline
    2 & & & 5 &  & \\\hline
    3 & & &6 &  & \\\hline
  \end{tabular}
\end{table*}
\begin{table*}[htbp]
  \centering
  \caption{The check points of tex.}\label{tab.list}
  \begin{tabular}{|c|c|c|c|c|}\hline
    \makebox[335pt][c]{Check Points}&\makebox[25pt][c]{Myself}&\makebox[25pt][c]{1}
    &\makebox[25pt][c]{2}&\makebox[25pt][c]{3}\\\hline
    % Check Points&myself&1&2&3\\\hline
    % *******************************************************************************
    % tex
    % *******************************************************************************
    題目の単語の頭文字は大文字(4文字未満の前置詞,冠詞は小文字)
    &\listcheck{1}{1}{0}{0}\\\hline
    苗字と名前の間は半角スペース
    &\listcheck{1}{1}{0}{0}\\\hline
    英語のフォントは Times New Roman
    &\listcheck{1}{1}{0}{0}\\\hline
    図やグラフを並べた際に図の大きさがそろっている
    &\listcheck{1}{1}{0}{0}\\\hline
    単位と数字の間にスペース 例:$\bigcirc$~10~mm,$\times$~10mm
    &\listcheck{2}{2}{0}{0}\\\hline
    単位は斜体にしない 例:$\bigcirc$~10~mm,$\times$~10~$mm$
    &\listcheck{1}{1}{0}{0}\\\hline
    変数に対する単位には[~]を付ける 例:$b$~[m], 5 m
    &\listcheck{1}{1}{0}{0}\\\hline
    文中や式中の変数の書体はイタリック,定数の書体はブロック
    &\listcheck{1}{1}{0}{0}\\\hline
    日本語文中の「(」,「)」,「,」,「.」は全角
    &\listcheck{1}{1}{0}{0}\\\hline
    % 一つ,二つ,三次元などは漢数字
    % &\listcheck{0}{0}{0}{0}\\\hline
    グラフ,文中の (a) などの括弧は半角
    &\listcheck{1}{1}{0}{0}\\\hline
    文中の数式関係の括弧は半角
    &\listcheck{1}{1}{0}{0}\\\hline
    比較検証の場合,同じ条件下での結果である
    &\listcheck{1}{1}{0}{0}\\\hline
    背景と目的がつながっている
    &\listcheck{1}{1}{0}{0}\\\hline
    省略されている単語の前に省略前の単語がある 例:Center of Mass(CoM)
    &\listcheck{1}{1}{0}{0}\\\hline
    参考文献の書き方が正しい
    &\listcheck{2}{2}{0}{0}\\\hline
    セミナー3日前に第一版のチェックを受けた
    &\listcheck{2}{2}{0}{0}\\\hline
    セミナー当日正午までに最終版のチェックを受けた
    &\listcheck{1}{1}{0}{0}\\\hline
    % *******************************************************************************
    % 参考文献
    % *******************************************************************************
    % 学会誌(注意:題名と学会誌名の後の「,」は全角)&&&&\\
    % 人名,``題名'',学会誌名,\verb*|vol. 2,no. 1,pp. 12--18, 2008.|
    % &-&&&\\\hline
    % 公演論文(日本の場合発表番号と年の間の「,」と文末の「.」だけ半角)&&&&\\
    % 人名,人名,``題名'',公演論文集名,発表番号\verb*|, 2009.|
    % &-&&&\\\hline
    % 卒論・修論(文末の「.」だけ半角)&&&&\\
    % 執筆者(記者名 \verb*|\ | 訳),``題目'',修士論文,&&&&\\
    % 東京都市大学工学部機械システム工学科,2009.
    % &-&&&\\\hline
    % 本(注意:「``」,「''」と文末の「.」以外は全角)&&&&\\
    % 著者名,``本の題名'',出版社,出版年,pp200--205.&&&&\\
    % (引用ページがない場合は「出版年.」)
    % &-&&&\\\hline
    % ホームページ(() 内は最後にそのページを見た日)&&&&\\
    % 作者名(わかる場合のみ).(\verb*|2010,\ Dec.\ 12|)ホームページ名&&&&\\
    % \verb*|\ [Online].\ Available:\url{|ホームページの URL\verb*|}|
    % &-&&&\\\hline
    % IEEEに基づく書き方例(題名,を「``」,「''」でくくる)&&&&\\
    % J. K. Author, ``Title of chapter in the book,''&&&&\\
    % in Title of His Published Book, xth ed. City of Publisher,&&&&\\
    % Country if not USA: Abbrev. of Publisher, year, ch. x,&&&&\\
    % sec. x, pp. xxx-xxx.&&&&\\
    % &-&&&\\\hline
  \end{tabular}
\end{table*}

\begin{table*}
  \centering
  \caption{The check points of ppt.}\label{tab.ppt}
  \begin{tabular}{|c|c|c|c|c|}\hline
    % *******************************************************************************
    % ppt
    % *******************************************************************************
    Check Points&\makebox[25pt][c]{Myself}&\makebox[25pt][c]{4}
    &\makebox[25pt][c]{5}&\makebox[25pt][c]{6}\\\hline

    日本語のフォントは MS~Pゴシック,英語は Times New Roman
    &\listcheck{0}{0}{0}{0}\\\hline
    真っ白なスライドデザインは使用しない
    &\listcheck{0}{0}{0}{0}\\\hline
    左右の余白が最低グリッド1マス分空いている
    &\listcheck{0}{0}{0}{0}\\\hline
    文字は単文にし,「.」は使用していない
    &\listcheck{0}{0}{0}{0}\\\hline
    文章の書き出しや,図表はページの中でそろえて配置
    &\listcheck{0}{0}{0}{0}\\\hline
    文字のサイズは四種類まで(大題,小題,本文,図のキャプション)
    &\listcheck{0}{0}{0}{0}\\\hline
    文字を見やすくするため,フォントサイズは 20 point 以上
    &\listcheck{0}{0}{0}{0}\\\hline
    % 強調のために赤と青の色を用いている (赤は瞬間的に,青は持続的に記憶される)
    % &\listcheck{0}{0}{0}{0}\\\hline
    矢印の色と位置がそろってる(色は単色とし,全てのスライドで同じ色を用いる)
    &\listcheck{0}{0}{0}{0}\\\hline
    写真,グラフ,表,動画を用いて見やすいスライドになっている
    &\listcheck{0}{0}{0}{0}\\\hline
    各スライドの発表時間が約30$\sim$60~秒になるような構成
    &\listcheck{0}{0}{0}{0}\\\hline
    「ご清聴ありがとうございました」を使っている(ご静聴ではない)
    &\listcheck{0}{0}{0}{0}\\\hline
    セミナー前にチェックを受けた
    &\listcheck{0}{0}{0}{0}\\\hline
  \end{tabular}
\end{table*}

%% Local Variables:
%% mode: japanese-latex
%% TeX-master: t
%% End:
